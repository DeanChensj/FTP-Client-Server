
\documentclass[11pt]{article} % use larger type; default would be 10pt
\usepackage{listings}
\usepackage[utf8]{inputenc} % set input encoding (not needed with XeLaTeX)
\usepackage{ctex}
%%% PAGE DIMENSIONS
\usepackage{geometry} % to change the page dimensions
\geometry{a4paper} % or letterpaper (US) or a5paper or....
% \geometry{margin=2in} % for example, change the margins to 2 inches all round
% \geometry{landscape} % set up the page for landscape
%   read geometry.pdf for detailed page layout information

\usepackage{graphicx} % support the \includegraphics command and options

% \usepackage[parfill]{parskip} % Activate to begin paragraphs with an empty line rather than an indent

%%% PACKAGES
\usepackage{booktabs} % for much better looking tables
\usepackage{array} % for better arrays (eg matrices) in maths
\usepackage{paralist} % very flexible & customisable lists (eg. enumerate/itemize, etc.)
\usepackage{verbatim} % adds environment for commenting out blocks of text & for better verbatim
\usepackage{subfig} % make it possible to include more than one captioned figure/table in a single float
% These packages are all incorporated in the memoir class to one degree or another...

%%% HEADERS & FOOTERS
\usepackage{fancyhdr} % This should be set AFTER setting up the page geometry
\pagestyle{fancy} % options: empty , plain , fancy
\renewcommand{\headrulewidth}{0pt} % customise the layout...
\lhead{}\chead{}\rhead{}
\lfoot{}\cfoot{\thepage}\rfoot{}

%%% SECTION TITLE APPEARANCE
\usepackage{sectsty}
\allsectionsfont{\sffamily\mdseries\upshape} % (See the fntguide.pdf for font help)
% (This matches ConTeXt defaults)

%%% ToC (table of contents) APPEARANCE
\usepackage[nottoc,notlof,notlot]{tocbibind} % Put the bibliography in the ToC
\usepackage[titles,subfigure]{tocloft} % Alter the style of the Table of Contents
\renewcommand{\cftsecfont}{\rmfamily\mdseries\upshape}
\renewcommand{\cftsecpagefont}{\rmfamily\mdseries\upshape} % No bold!

%%% END Article customizations

%%% The "real" document content comes below...

\title{Computer Networking\\ Assignment 2 report}
\author{5120309491 Chen Shangjie}
\date{} % Activate to display a given date or no date (if empty),
         % otherwise the current date is printed

\begin{document}
\maketitle

\section{Description \& Runtime environment}
I implement the project in C Language and the whole system is working on Linux System.
User can easily use my ftp client and server by command-line operations.
The interface for both local ftp server and the real ftp server are the same, i.e,
you can access to the real ftp server freely, as well.

\section{Build \& Run}
You may use 'cd' to move to the prj directory and type 'make' to build the whole prj.
The executable will be located at  \verb| bin/client|
  and  \verb| bin/server.|

\textbf{ !!! Make sure you have already create 'bin/server', 'bin/client', and 'obj' those three directory 
in the current directory.  } 

Use \verb|./S <port number> |  and  \verb| ./C <host name> <port number> | to
run the programm.

\section{Command and detailed description}
\subsection{Commands you may use:}
\begin{itemize}
\item\ {USER $<user name>$} (\verb|not necessary for local|)
\item\ {PASS $<password>$} (\verb|not necessary for local|)
\item\ {PASV} (\verb|to enter the passive mode for transmission|)
\item\ {NLST} (\verb|list all the files and directory|)
\item\ {DELE $<file name>$} (\verb|delete the file|)
\item\ {RETR $<file name>$} (\verb|download file from the ftp|)
\item\ {STOR $<file name>$} (\verb|upload file to the ftp|)
\item\ {CWD $<dir name>$} (\verb|change the current directory|)
\item\ {BYE} (\verb|change the current directory|)
\end{itemize}
Other commands will not be received by the server, the server will response 500.
\subsection{Implementation on local server}
\subsubsection*{NLST:}
Invoke $system()$ the system call $ls$ and load the result to a tmp file,
read the file and then unlink the tmp file.
\subsubsection*{DELE:}
Invoke $remove()$ to remove the file directly.
\subsubsection*{CWD:}
Invoke $chdir()$ to change the current directory. if the return value equals to 0,
invoke $getcwd()$ to get the current directory.
\subsubsection*{RETR \& STOR:}
According to the FTP protocol, when there need a data transmission, we need to create another
data socket to transfer file. The user should send the 'PASV' command first to change the mode,
then receive the new port number provided by the server, the client will connect to the new port
and do the transfer. At this time, use $fread()$ and $fwrite()$ to do the file operation, continue write
data which received from the data socket to the file.
The detailed code will not be covered in this report.

\subsection{FTP protocol related}
\subsubsection{FTP reply codes}
Below is a summary of FTP reply codes that may be returned by an FTP server.\\
The first digit is used to indicate one of three possible outcomes �� success, failure, or to indicate an error or incomplete reply:
\begin{itemize}
\item\ 2yz �C Success reply
\item\ 4yz or 5yz �C Failure reply
\item\ 1yz or 3yz �C Error or Incomplete reply
\end{itemize}
The second digit defines the kind of error:
\begin{itemize}
\item\ x0z �C Syntax. These replies refer to syntax errors.
\item\ x1z �C Information. Replies to requests for information.
\item\ x2z �C Connections. Replies referring to the control and data connections.
\item\ x3z �C Authentication and accounting. Replies for the login process and accounting procedures.
\item\ x4z �C Not defined.
\item\ x5z �C File system. These replies relay status codes from the server file system.
\end{itemize}
The third digit of the reply code is used to provide additional detail for each of the categories defined by the second digit.
\subsubsection{About commands}
Each command must be ended up with \verb|\r\n| (no bother to type it the program will solve it),  the specific commands can be found
in the internet...

\section{Socket Programming}
The main process of the socket connection is showed in Figure 1.
\begin{figure}\centering
 \includegraphics[width=0.8\linewidth]{socket.jpg}
\caption{Socket}
\label{fig:res}
\end{figure}

\section{Trouble shooting \& experience}
\subsection{About the receive command}
There are two types of received command from the FTP server, one of them
begin with number and space, such as '220 ', and then some words, finished with \verb|\r\n.|
The other begin with number and '-', such as '220-' ,which means there still exist some more information
to transfer to the client.

In order to analyze the received command, I put all the received message into one buffer,
use $strtok()$ split it into pieces by \verb|'\n'. |
Judge every piece of command whether the fourth char of it is space, if true, end of $recv()$.

The detailed implementation is in function $parse()$ within C.c .

\subsection{About the data port}
When the client send the PASV to the server, the server will send the new data port.
The return information will be like this:

227 Entering Passive Mode (h1,h2,h3,h4,p1,p2)

The first four number are the IP address, then the port number. We can
get the port number by $p1 * 256 + p2$ .

\subsection{About the type}
Some kind of file may not be correctly download, we need to change the type of FTP to binary,
so that the format won't be a problem so that I encapsulate the TYPE command in the PASV command.
\subsection{Some Comments}
Working on this assignment is challenging though interesting, we can work on real FTP protocol
which helps us understanding the SOCKET and FTP better.  After all the job is done, I really gains
a lot.

Thanks for Prof.Shen and TA's work, thanks for my classmate's help.

\section{Screenshot}
\begin{figure}\centering
 \includegraphics[width=0.8\linewidth]{ftp.png}
\caption{Connect with FTP server (1)}
\end{figure}

\begin{figure}\centering
 \includegraphics[width=0.8\linewidth]{ftp2.png}
\caption{Connect with FTP server (2)}
\end{figure}

\begin{figure}\centering
 \includegraphics[width=0.8\linewidth]{local1.png}
\caption{Connect with local server (1)}
\end{figure}

\begin{figure}\centering
 \includegraphics[width=0.8\linewidth]{local2.png}
\caption{Connect with local server (2)}
\end{figure}

\begin{figure}\centering
 \includegraphics[width=0.8\linewidth]{local3.png}
\caption{The local server}
\end{figure}

\end{document}
